% !TeX root = ../thuthesis-example.tex

\chapter{遗忘方法的实现与验证}

\section{实验验证}

\subsection{实验环境介绍}
我们用到的实验设备是两台服务器,
\\cpu Intel(R) 
\\Core(TM) i9-9900K CPU @ 3.60GHz
\\Intel(R) Core(TM) i7-6700K CPU @ 4.00GHz
\\内存 32g 32g
\\硬盘 ssd ssd
\\显卡  1xNvidia Geforce 2080 3xNvidia Geforce 1080
\\操作系统 ubuntu16.04LTS ubuntu16.04LTS
\\深度学习框架pytorch
\\数据集是cifar-10,正常训练集50000张图片,测试集10000张图片。遗忘两个类别,遗忘集是从正常训练集分离出来的10000张图片,遗忘测试集2000张图片。保留训练集是指正常训练集出去了遗忘集外的数据集合。保留训练集40000张图片,保留测试集8000张图片。神经网络有Resnet18和Resnet50。
\subsection{实验设计}

\subsubsection{确定冻结层数实验}
实验一:确定冻结层数实验
\\实验目的:这个实验的目的是确定网络冻结的层次数。确定的标准是根据3.5中讲到的四个指标的综合指标。
\\实验准备:正常训练集,遗忘集和保留集,还有测试集。使用的神经网络框架是Resnet18。
\\实验过程:首先使用正常训练集去训练神经网络,直至训练集准确率收敛,获得模型1,在训练网络之前保存神经网络训练之前的网络参数,记为模型0。再使用保留集重新训练一个神经网络,获得模型2。将模型1的最后一层(全联接层)的参数替换为模型0的最后一层参数,模型1其余层数的参数不变,由此得到模型1\_reset\_fc\_before\_training。将模型1\_reset\_fc\_before\_training加载到一个新的神经网络中,使用保留集去训练这个新的网络,直至训练准确率收敛,在训练过程中保持除全连接层以外层次的参数不被更新(即冻结),得到模型1\_reset\_fc\_after\_training。将模型1\_reset\_fc\_after\_training加载到模型,分别测量并记录指标一、指标二、指标三和指标四。
\\实验结束:计算并记录综合指标计算结果。
\subsubsection{冻结必要性验证实验}
实验目的:通过本实验验证冻结较低层次的参数是否对加快遗忘训练收敛速度有一定的贡献,同时观察遗忘集集准确率、保留集准确率和测试集准确率的变化情况。
\\实验准备:数据集有正常训练集,保留集。神经网络是Resnet18。
\\实验过程:
\\1. 用正常训练集训练神经网络得到模型1
\\2. 在正常训练集训练之前保留网络参数得到模型0
\\3. 用模型0的全连接层参数,替换模型1的全连接层的参数,得到模型1\_reset\_fc\_before\_training
\\4. 用神经网络加载模型1\_reset\_fc\_before\_training,网络参数全部无需冻结,用保留集训练至训练准确率收敛,得到模型1\_reset\_fc\_after\_training
\\5. 分别记录下各个指标
\\6. 分别将神经网络从最后一层到各个层的参数重置后,重复上述3-5步骤,将全连接层替换成全连接层至各个层之间的参数
\\实验结束:计算并记录各个模型综合指标计算结果
\subsubsection{反向冻结验证实验}
实验目的:验证卷积神经网络的分层抽象特性,与正向冻结实验进行对照。
\\实验准备:数据集有正常训练集,保留集。神经网络是Resnet18。
\\实验过程:
\\1. 用正常训练集训练神经网络得到模型1
\\2. 在正常训练集训练之前保留网络参数得到模型0
\\3. 用模型0的第一卷积层参数,替换模型1的第一卷积层参数,得到模型1\_reset\_conv1\_before\_reverse\_training
\\4. 用神经网络加载模型1\_reset\_conv1\_before\_reverse\_training,除了第一卷积层外,其余层数的参数全部冻结。用保留集训练至训练准确率收敛,得到模型1\_reset\_conv1\_after\_reverse\_training
\\5. 分别记录下各个指标
\\6. 分别重置第一卷积层至最后一层全连接层参数,重复上述第3-5步骤。
\\实验结束:计算并记录各个模型综合指标计算结果
\subsubsection{遗忘可持续性验证实验}
实验目的:检验冻结重置方法随着遗忘类别数量的增多有效性的变化情况
\\实验准备:数据集有正常训练集,分别遗忘掉1-9个类别的保留集1-保留集9共9个保留集。神经网络是Resnet18。
\\实验过程:
\\1. 用正常训练集训练神经网络得到模型1
\\2. 在正常训练集训练之前保留网络参数得到模型0
\\3. 使用保留集1-保留集9分别重新训练模型,得到模型\_forget\_1\_retrain,...,模型\_forget\_9\_retrain。
\\4. 用神经网络加载模型1,然后把神经网络全连接层和6层卷积层的参数重置为模型0的相应层次的参数。
\\5. 分别用保留集1-保留集9去训练步骤4中生成的网络,训练过程中冻结除了全连接层和后6层卷积层的参数。得到模型1\_fc\_conv6\_retain\_1\_finetune,...,模型1\_fc\_conv6\_retain\_9\_finetune共9个神经网络的参数。
\\6. 5中得到的网络参数进行指标测试,并记录。
\\实验结束:计算并记录各个模型综合指标的计算结果
\section{实验结果}

\section{本章小结}

